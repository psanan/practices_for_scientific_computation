\documentclass{beamer}
\usepackage{algorithm}
\usepackage[noend]{algpseudocode}
\usepackage{listings}
\usepackage{xcolor}
\definecolor{verylightgray}{gray}{0.95}
\definecolor{somewhatdarkgray}{gray}{0.3}
\lstset{
  language=,
  basicstyle=\footnotesize\ttfamily,
  numberstyle=\tiny,
  numbersep=5pt,
  tabsize=2,
  extendedchars=true,
  breaklines=true,
  keywordstyle=\color{red},
  stringstyle=\color{blue}\ttfamily,
  numberstyle=\color{violet},
  showspaces=false,
  showtabs=false,
  showstringspaces=false,
  backgroundcolor=\color{verylightgray},
  frame=single,
  framexleftmargin=3px,
  framexrightmargin=3px,
  rulecolor=\color{lightgray},
  basewidth=0.5em,                % For consistent spacing with inline listings
  columns=flexible,               % For copy-pasting
}
\usepackage{color}
\usepackage{amsmath}

\hypersetup{linkcolor=,urlcolor=magenta,colorlinks=true}

\usetheme{Madrid}
\setbeamertemplate{navigation symbols}{} % Remove the navigation symbols
\setbeamertemplate{sections/subsections in toc}[default] %Turn off ugly toc numbering
\setbeamertemplate{itemize subitem}[triangle]
\setbeamertemplate{itemize item}[triangle]
\setbeamertemplate{enumerate items}[default]
%\setbeamercovered{transparent}
\usepackage{appendixnumberbeamer}
\defbeamertemplate{section page}{customsection}[1][]{%
  \begin{centering}
    {\usebeamerfont{section name}\usebeamercolor[fg]{section name}#1}
    \vskip1em\par
    \begin{beamercolorbox}[sep=12pt,center]{part title}
      \usebeamerfont{section title}\insertsection\par
    \end{beamercolorbox}
  \end{centering}
}
\defbeamertemplate{subsection page}{customsubsection}[1][]{%
  \begin{centering}
    {\usebeamerfont{subsection name}\usebeamercolor[fg]{subsection name}#1}
    \vskip1em\par
    \begin{beamercolorbox}[sep=8pt,center,#1]{part title}
      \usebeamerfont{subsection title}\insertsubsection\par
    \end{beamercolorbox}
  \end{centering}
}
\setbeamertemplate{section page}[customsection]
\setbeamertemplate{subsection page}[customsubsection]
\AtBeginSection{\frame{\sectionpage}}
\AtBeginSubsection{\frame{\subsectionpage}}

\graphicspath{{./images/}}

\usepackage{booktabs}

\author{Patrick Sanan}
\institute[ETHZ]
{
Institute of Geophysics, ETH Zurich\\
\href{mailto:patrick.sanan@erdw.ethz.ch}{patrick.sanan@erdw.ethz.ch}
}


\title{Practices for Scientific Computation}
\subtitle[]{Principles, Corollaries, and Simple Applications}
\date[]{}
% Long date messes up footer

\begin{document}


\begin{frame}[fragile]
\titlepage
\begin{center}
\href{https://github.com/psanan/practices_for_scientific_computation}{https://github.com/psanan/practices\_for\_scientific\_computation}
\end{center}
\end{frame}

\begin{frame}
\tableofcontents
\end{frame}

\section{Setting}

\begin{frame}[fragile]
\frametitle{Motivation}
\begin{itemize}
\item Scientists are increasingly relying on computational tools, and spending their time struggling with them.
\item People often repeat the same mistakes as they develop workflows by trial and error, and reach the same ultimate conclusions
\item Later, I will tell you that you need to learn about version control (git). Why should you bother?
\end{itemize}
\end{frame}

\begin{frame}[fragile]
\frametitle{Assumed Audience}
\begin{itemize}
\item You're a scientist who works with numerical, computational tools (simulation, data processing and analysis), beyond the (exclusive) use of commercial packages
\item You know how to work from the command line
\item You're interested in using the available tools more efficiently
\end{itemize}
\end{frame}

\begin{frame}[fragile]
\frametitle{Outline and Scope}
\begin{itemize}
  \item I will attempt to provide you with a few principles that I feel that I (and those I work with) have ``learnined the hard  way''
  \item I will presnet a few concrete applications of these principles, focusing on common tasks often considered boring.
\end{itemize}
\end{frame}

\begin{frame}[fragile]
\frametitle{Beating the ``Greedy Algorithm''}
\begin{itemize}
  \item Smart people working quickly tend to make similar choices when presented with the available tools
\item Often, these work well, but not always, because they make ``locally optimal'', not ``globally optimal'' choces
\item The principles that follow are about adding rules to try to avoid getting trapped in ``local minima''.
\end{itemize}
\end{frame}

\section{Principles}

\subsection{``You will forget''}
\begin{frame}[fragile]
\frametitle{Principle: ``You will forget''}
\begin{itemize}
\item After a given project leaves your immediate attention, you will remember few of the technical details
\item The Greedy Algorithm:
\begin{itemize}
\item Only consider the exact, current use case
\item Don't write much down
\end{itemize}
\item Observations
\begin{itemize}
\item Your own code, from a few months ago, looks like a stranger's code
\item Context switching has large overheads
\item ``Getting it to run again'' is often a very time-consuming and inspiration-sucking exercise.
\end{itemize}
\end{itemize}
\end{frame}

\begin{frame}[fragile]
\frametitle{Corollaries: ``You will forget''}
  \begin{itemize}
    \item State is your enemy
    \item Encapsulation is essential
    \item Document, but avoid documentation which you have to remember to update
     %\item Avoid technical debt
  \end{itemize}
\end{frame}

\subsection{``Good practices are good science''}
\begin{frame}[fragile]
\frametitle{Principle: ``Good practices are Good Science''}
\begin{itemize}
  \item Treating computational tools as first-class parts of the scientific process is effective in increasing the quality of scientific output.
\item The Greedy Algorithm
\begin{itemize}
\item Treat computational methods as simply another tool in the toolbox, applying when needed
\item When presenting results, consider computational details to be uninteresting and trivial
\end{itemize}
\item Observations
\begin{itemize}
\item The ``Third Pillar of Science'' concept is real
\item It can be surprisingly difficult to reproduce computational results (even your own)
\item Applying scientific ethics to the use of computational tools can be effective
\end{itemize}
\end{itemize}
\end{frame}

\begin{frame}[fragile]
\frametitle{Corollaries: ``Good practices are Good Science''}
\begin{itemize}
\item Challenge your code like you challenge ideas: try to prove it wrong. Get rid of things that don't work.
\item The value of clarity, reproducibility and transparency extends to your code
\item Document your work as you progress
\end{itemize}
\end{frame}

\subsection{``Numerical codes are special''}
\begin{frame}[fragile]
\frametitle{Principle: ``Numerical codes are special''}
\begin{itemize}
  \item Numerical codes present distinct challenges compared to both general computation / computer science, and general scientific endeavors; these need to be dealt with directly.
\item The Greedy Algorithm:
\begin{itemize}
\item If it seems to work, good enough
\end{itemize}
\item Observations
\begin{itemize}
\item Floating-point arithmetic, particularly in parallel, requires special thought
\item Seemingly-simple code can rely on a great deal of mathematical or physical expertise
\item Numerical methods are often not robust to seemingly-minor changes in the problem setting
\end{itemize}
\end{itemize}
\end{frame}

\begin{frame}[fragile]
\frametitle{Corollaries: ``Numerical codes are special''}
\begin{itemize}
  \item ``If you see something strange- there is a bug!!''\footnote{Taras Gerya} % TODO put in the actual ref (pdsbib repo as submodule)
  \item It pays to learn the ``Numerical facts of life''\footnote{Ed Beuler, ``PETSc for Partial Differential Equations'' (to be published by SIAM Press)}, for example the fundamental limits of finite-precision arithmetic.
  \item Numerical codes need (easy-to-run) tests
  % \item ``Don't believe it until you can run it''
%\item Don't vary more than one at a time: physics, numerical method, code logic, or computational system.
\end{itemize}
\end{frame}

\subsection{``Less.'}
\begin{frame}[fragile]
\frametitle{Principle : ``Less.''}
\begin{itemize}
\item
\item The Greedy Algorithm
\begin{itemize}
\item Copy-paste instead of encapsulating and re-using
\item Multiple copies of the same data
\item Leave commented-out code, just in case
\item Add to the same file until it's 1000s of lines long
\end{itemize}
\item Observations
\begin{itemize}
\item It's easier to write code than read it
\item Unforeseen problems and difficult bugs become much more prevalent, the more complicated a system is
\item Code ``rots''
\item Hierarchy, encapsulation and data-hiding make big things small
\end{itemize}
\end{itemize}
\end{frame}

\begin{frame}[fragile]
\frametitle{Corollaries: ``Less.''}
\begin{itemize}
\item Treat code as a liability, as much as an asset
\item Look out for ways to delete and encapsulate things
\item Try to present work to other people in chunks of less than a ``screen''.
\end{itemize}
\end{frame}

\section{Practices}

\begin{frame}[fragile]
How can we apply our principles?
\begin{enumerate}
\item``You will forget''
\item``Good practices are good science''
\item``Numerical codes are special''
\item``Less.''
\end{enumerate}
\end{frame}

\subsection{The art of the README}
\begin{frame}[fragile]
\frametitle{Practice: The art of the README}
  \begin{itemize}
    \item This is your one and only change to say something to people when they first come across your project.
    \item A quickstart is incredibly valuable
      \begin{itemize}
        \item To goal is to get allow a user to have a working state to start from
        \item Show, don't explain
        \item Have copy-paste-eable commands (test them, exactly, on more than one system)
      \end{itemize}
    \end{itemize}
\end{frame}

\subsection{Good login files}
\begin{frame}[fragile]
\frametitle{Practice: Good login files}
\begin{itemize}
  \item Avoid state
  \item Be very careful about adding things to \lstinline{PATH}, \lstinline{PYTHONPATH}, \lstinline{LD_DYNAMIC_LIBRARY_PATH}, etc.
  \item Do not assume that you will only work on one project.
  \item Explicit is better than implicit
  \item Use a single login file on all systems
\end{itemize}
\end{frame}

% TODO finish
\subsection{Asking good technical questions}
\begin{frame}[fragile]
\frametitle{Practice: Asking good technical questions}
  \begin{itemize}
    \item ``You will forget'': You will very likely come back to this question and answer again, later. Write it clearly.

\item ``Less'':
  a. the XY problem
      b. Concisely describing the actual problem very often leads you to the answer, yourself (``rubber ducking'')
  c. Confine long output to attachments
\item ``Numerical code is special'': If the problem isn't reproducible, or you don't have  avery similar working state, it's very hard to diagnose problems.
  d. Bisectability is the key
  e. Value the recipient's time
  \end{itemize}

  For more, see \href{https://blogs.egu.eu/divisions/gd/2019/07/03/it-doesnt-work-asking-questions-about-scientific-software/}{my post, ``It doesn't work!'', on the EGU blog}.
\end{frame}

\subsection{Using version control}
\begin{frame}[fragile]
\frametitle{Practice: Using version control}
Good use of version control helps satisfy all of our principles:
  \begin{itemize}
    \item ``You will forget'': keep track of past states, build repositories with documentation
    \item ``Good practices are good science'': Identify exact versions of software, distribute code, collaborate on code.
    \item ``Numerical codes are special'': Manage working states effectively, easily move to new machines
    \item ``Less'': maintain a single, canonical history of a project.
  \end{itemize}
  For these and many other reasons, there is \href{https://github.com/psanan/git_tutorial}{an entire git tutorial available at github.com/psanan/git\_tutorial}.
\end{frame}

\end{document}
